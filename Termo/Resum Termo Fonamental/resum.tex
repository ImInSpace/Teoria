\documentclass[a4paper]{article}
\usepackage[margin=2cm]{geometry}
\usepackage{fontspec}

\usepackage[catalan]{babel}
\usepackage{csquotes}

\setlength{\parindent}{0pt}
\setlength{\parskip}{1em}

\title{Resum Termodinàmica Fonamental}
\author{Joan Marcè Igual}

\begin{document}
\maketitle

\section{Primer principi de la termodinàmica}

El primer principi de la termodinàmica reflexa la \emph{Llei de conservació de l'energia}. Es formula de la següent manera:

\begin{displayquote}
	\em
	En un sistema adiabàtic (que no hi ha intercanvi de calor amb altres sistemes o el seu entorn) que evoluciona d'un estat ideal $\mathcal{A}$ a un estat final $\mathcal{B}$, el treball realitzat no depèn ni del tipus de treball ni del procés seguit.
\end{displayquote}

Així doncs per un sistema tancat:

$$
\Delta U = Q - W
$$

On:
\begin{description}
	\item[\boldmath $\Delta U$] és la variació d'energia del sistema
	\item[\boldmath $Q$] és la calor intercanviada pel sistema a través de les parets
	\item[\boldmath $W$] és el treball intercanviat pel sistema amb el seu entorn
\end{description}

\end{document}