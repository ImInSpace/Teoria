\documentclass[a4paper]{article}

\usepackage{amsmath,amssymb,amsfonts,amsthm}
\usepackage[catalan]{babel} % Language 
\usepackage{fontspec}
\usepackage[margin=2cm]{geometry}

\setlength{\parindent}{0pt}
\setlength{\parskip}{1em}

\newcommand{\verteq}{\rotatebox{90}{$\,=$}}

\begin{document}
	\title{Tema 2: Visualització de dades}
	\maketitle
	
	\section{Introducció}
	
	Antigament les dades eren molt fàcils, l'estadístic mateix podia anar a qualsevol lloc i mesurar el que li interessés. Avui en dia amb el \emph{Big Data} això és força diferent. Tot i així en tots dos casos les dades solen tenir més de dues dimensions. 
	
	\emph{Fischer} (estadístic famós) fa un temps va recollir una sèrie de dades sobre peixos, va obtenir 4 tipus de dades diferents cosa que feia difícil la visualització simultània d'aquestes dades.
	
	\section{Anàlisi de Components Principals (PCA)}
	
	\textbf{Objectiu}: reduir la dimensió preservant la major part de la informació \emph{rellevant} (variància). 
	
	\textbf{Exemple}: es vol projectar un llapis (objecte en 3D) en un pla (2D).
	
	Podem escollir diversos plans on projectar el llapis. Si es fa una projecció en alçat es veurà el llapis estirat i segons com es projecti de perfil només es veurà un cercle.
	
	Quina de les dues representacions és millor? Què volem? 
	\begin{itemize}
		\item Representar les dades de la manera més fidel? (Representació en alçat)
		\item Representar les dades de la manera més reduïda possible? (Representació de perfil)
	\end{itemize}
	
	També es poden representar les dades sobre l'eix de revolució del llapis. O sobre un eix perpendicular a aquest. El que busquem és que les dades tinguin la màxima variabilitat possible.
	
	\textbf{Formalització}: Tenim una mostra de dades $\{x_1, ..., x_n\}$ tal que $x_n \in \mathbb{R}^d$; provenen d'un vector aleatori $X = (X_1, ..., X_d)^T$ que té mitjana $\mu \in \mathbb{R}^d$ i una matriu de covariàncies $\Sigma = \sigma_{ij}^2$.
	
	\begin{itemize}
		\item $\mathbb{E}(X) = \mathbb{E}(X_1, ..., X_d) = \mu$
		\item $\mathbb{E}((X - \mu)(X-\mu)^T) = \Sigma$
		
		En particular $CoVar(X_i, X_j) = \sigma_{ij}^2$
		
		$CoVar(X_i, X_i) = Var(X_i) = \sigma_{ii}^2=\sigma_i^2$
	\end{itemize}
	
	Considerem el problema de trobar unes variables noves $Y = (Y_1, ..., Y_d)^T$ tal que:
	
	\begin{enumerate}
		\item $CoVar(Y_i, Y_j) = 0$ per $i \ne j$
		\item $Var(Y_1) > Var(Y_2) > ... > Var(Y_d)$
		\item $\sum_{i=1}^d Var(X_i) = \sum_{i=1}^d Var(Y_i)$
	\end{enumerate}
	
	Busquem un \textbf{mètode lineal}: les $Y_j = a_j^TX, j=1,...,d$
	
	Imposem una condició de \textbf{normalització} $||a_j||^2 = 1$ 
	(Transformació \textbf{ortogonal})
	
	\subsection{Tria de $a_1$}
	Cal que $a_1$ maximitzi la $Var(Y_1)$ subjecte a $||a||^=1$
	\[Var(Y_1) = Var(a_1^TX) = a_1^TVar(X)a_1 = a_1^T\Sigma a_1\]
	
	\textbf{Excursió}: multiplicadors de Lagrange
	
	Sigui $f:\mathbb{R}^d \rightarrow \mathbb{R}$ diferenciable i la volem maximitzar subjecte a una \textbf{condició d'igualtat} $g(x_1, ..., x_d) = c$.
	
	Una manera de fer-ho és construir el Lagrangià:
	
	\[ L(x) = f(x) - \lambda(g(x) - c) \]
	
	La solució és un punt estacionari (les derivades s'anu\lgem en).
	\[ \frac{\partial L}{\partial x_i} - \lambda \frac{\partial g}{\partial x_i} = 0, i = 1, ..., d \implies L(a_1) = a_1^T\Sigma a_1 - \lambda(||a||^2 - 1) \]
	
	\begin{itemize}
		\item $\frac{\partial L}{\partial a_1} = 2\Sigma a_1 - 2\lambda a1 = 0$
		\item $\Sigma a_1 = \lambda a_1$
		
		$a_1$ és un vector propi de $\Sigma$ amb valor propi $\lambda$
		\item $Var(Y_1) = a_1^T\Sigma a_1=a_1^T(\lambda a_1) = \lambda(a_1^T a_1) = \lambda ||a_1||^2 = \lambda$
		
		\textbf{Mètode}: triem $\lambda$ on el VAP $\Sigma$ més gran
		
		$\lambda_1, ..., \lambda_d \implies \lambda_1 > \lambda_2 > ... > \lambda_d > 0$
	\end{itemize}
	
	\subsection{Tria de $a_2$}
	
	$Y_2 = a_2^TX$ on $||a_2||^2 = 1$ i \[CoVar(Y_2, Y_1) = CoVar(a_2^TX, a_1^TX) = a_2^T\Sigma a_1 \underbrace{= 0}_\text{Volem que sigui 0}\] 
	
	\[ = a_2^T(\Sigma a_1) = a_2^T(\lambda_1 a_1) = \lambda_1 (a_2^Ta_1) \underbrace{= 0}_\text{volem que ho sigui} \Leftrightarrow a_1 \bot a_2 \]
	
	\[ L(a_2) = a_2^T\Sigma a_2 - \lambda(||a_2||^2 - 1) - \delta(a_2^Ta_1) \implies \Sigma a_2 = \lambda a_2 \]
	
	Triem $a_2$ com el vector associat al segon VAP més gran de $\Sigma \rightarrow \lambda_2$
	
	\subsection{Resultats}
	\begin{enumerate}
		\item Si diem $\Delta$ a la matriu de covariàncies de $Y$, és clar que
		\[ \Delta = 
		\begin{pmatrix}
		\lambda_1 & 0 & \cdots & 0 \\
		0 & \lambda_2 & \ddots  & \vdots \\
		\vdots & \ddots & \ddots & 0 \\
		0 & \cdots & 0 &  \lambda_d
		\end{pmatrix} \]
		
		donats: $a_3, ..., a_d$ es deriven igual els $a_1, ..., a_d$ es diuen \textbf{components principals}.
		
		\item Per construcció
		\begin{equation}			
			\overset{Var(Y_1)\ >}{\underset{\lambda_1}{\verteq}} \overset{Var(Y_2)\ >}{\underset{\lambda_2}{\verteq}}
			\overset{...\ >\ Var(Y_d)}{\underset{\lambda_d}{\verteq}}
		\end{equation}
		 
		
		\item $\sum_{i=1}^d Var(Y_i) = \sum_{i=1}^d \lambda_i \underbrace{=}_\text{teorema} Tr(\Sigma)$
		
		$ = \sum_{i=1}^d \sigma_i^2 = \sum_{i=1}^d Var(X_i)$
	\end{enumerate}
	
	\textbf{Idea} quedar-se amb els $k$ primers components principals $a_1, ..., a_k (k < d)$
	
	Si decidim quedar-nos amb els k primers 
	
	% Formula 1 aquí
	
	\textbf{Algorisme} PCA $(X, k = 2)$
	
	\begin{itemize}
		\item Calcular la mitjana de les dades $\hat{\mu}$
		\item Centrar les dades $x_n \leftarrow x_n - \hat{\mu}, n = 1, ..., N$
		\item Calcular els VEPs i VAPs de $\hat{\Sigma}$
		
		ret $A \leftarrow (a_1, a_2, ..., a_k)$
	\end{itemize}
	
	\section{Anàlisi discriminiant de Fisher}
	\[ \mathcal{D} = \{(x_1, t_1), ..., (x_N, t_N)\}, x_n \in \mathbb{R}^d, t_n \in \{-1, +1 \} \]
	
	% Figura 2
	
	\textbf{Notació}
	
	\[ c_+, c_- \]
	
	\begin{equation}
		\begin{aligned}
			c_+ &:= \{ n  t_n = +1 \} \\
			c_- &:= \{ n | t_n = -1 \} \\
			m_+ &:= \frac{1}{|c_+|} \sum_{n\in c_+} x_n \\
			m_- &:= \frac{1}{|c_-|} \sum_{n\in c_-} x_n \\
		\end{aligned}
	\end{equation}
	
	Busquem un $w$ tal que maximitzi la separació entre les mitjanes projectades $|w^Tm_+ - w^Tm_-| = |w^T(m_+ - m_-)|$
	
	La \textbf{dispersió} (scatter) d'una classe projectada:
	
	\[ x_+^2 := \sum_{n \in c_+} (w^Tx_n - w^Tm_+)^2 \]
	(és la variància, excepta la divisió per $|c_+| - 1$)

	$s_-^2$ anàlogament
	
	La \textbf{dispersió total} és $s_+^2 + s_-^2$.
	
	El criteri de Fisher és \[ J(w) = \underbrace{\frac{|w^T(m_+-m_-)|}{s_+^2 + s_-^2}}_\text{a maximitzar} \]
	
	La solució és $w^* = S_w^-1(m_- - m_+)$
	
	$S_w$ és la matriu de dispersió intra-classe (within-class scatter matrix).
	
	\[ S_w = \sum_{n \in c_+} (x_n - m_+)(x_n - m_+)^T + \sum_{n \in c_-} (x_n - m_-)(x_n - m_-)^T \]
	
\end{document}