\documentclass[a4paper]{article}

\usepackage{amsmath}
\usepackage{amssymb,amsfonts}
\usepackage[catalan]{babel} % Language 
\usepackage{fontspec}
\usepackage[margin=2cm]{geometry}
\usepackage{graphicx}

\setlength{\parindent}{0pt}
\setlength{\parskip}{0.2cm}

\newcommand{\verteq}{\rotatebox{90}{$\,=$}}

\title{Tema 3: Mètodes de \emph{clustering}}

\begin{document}
	
\maketitle

\textbf{Objectiu}: El nostre objectiu és trobar agrupacions naturals de dades. Un grup o subgrup de dades és un \emph{cluster}

Les observacions que pertanyen al mateix \emph{cluster} s'assemblen entre elles

Les observacions que \textbf{no} pertanyen al mateix \emph{cluster} no s'assemblen tant

\textbf{\emph{Clustering:}}
\begin{itemize}
	\item el \textbf{procés} de trobar els \emph{clusters} presents en les dades
	\item el \textbf{resultat} del procés.
\end{itemize}

% Figures 1 i 2

\section{Mètodes de \emph{clustering}}

Un tipus de mètode són els mètodes jeràrquics:
\begin{itemize}
	\item \textbf{divisius} es dediquen a separar les dades en dos grups recursivament fins que només queda un punt.
	\item \textbf{aglomeratius} es dediquen a agrupar les dades des d'un punt fins a obtenir el \emph{cluster} final.
\end{itemize}

% Figura 3

També hi ha els mètodes combinatoris. Hi ha un algoritme voraç amb una funció de qualitat.

Hi ha els algoritmes probabilístics. De quantes maneres es poden agrupar $N$ dades en $K$ \emph{clusters}?

$$ S(N, K) = \frac{1}{K!} \sum_{k=1}^K (-1)^{K-k} \begin{pmatrix}
K \\ k
\end{pmatrix} \text{, nº d'Stirling del 2n tipus} \implies S(19,4) \simeq 10^{10} $$

$$ \underbrace{B(N)}_{\text{nº de Bell}} \sum_{K=1}^{N} S(N,K) \implies \text{ ex.: } B(79) \simeq 3.89·10^85 $$

Així doncs hi ha diferents mètodes de \emph{clustering} per trobar els grups.

\section{Algorisme de $k$-means}

Tenim una mostra $\mathcal{D} = \{ x_1,..., x_N \} , x_i \in \mathbb{R}, 1 \le i \le N$. Fixa't $K$ externament, escollim un conjunt de $K$ \textbf{prototips}:

$$ \mathcal{P} = \{ \mu_1, ..., \mu_K \}, \mu \in \mathbb{R}^d $$

% Figura 4

\begin{itemize}
	\item El nostre objectiu és trobar un conjunt de prototips $\mathcal{P}$ tal que les dades de $\mathcal{D}$ que estiguin assignades al \emph{cluster} $k$ es trobin més a prop de $\mu_k$ que de cap altre.
	\item Definim les variables indicadores:
	
	$$ r_{nk} =
	\begin{cases}
	1 & \text{si } k = argmin ||x_n - \mu_j||\ 1 \le j \le K \\ 0 & \text{en cas contrari}
	\end{cases}$$
	
	\item Definim un criteri de qualitat, a minimitzar, tenint en compte que minimitzar la distància al quadrat és el mateix que minimitzar la distància:
	
	$$ J(\mathcal{P}, \{r_{nk}\}) = \sum_{n=1}^N \sum_{k=1}^K r_{nk} ||x_n - \mu_k||^2 $$
	
	\begin{itemize}
		\item Si sapiguessim la solució per $\mathcal{P}$, com trobaríem la solució pells $\{r_{nk}\}$? Calculant les distàncies als $\mathcal{P}$.
		
		\item Si sapiguessim la solució per $\{r_{nk}\}$ com trobaríem la solució per $\mathcal{P}$? El prototip òptim és el centroide (la mitjana) de les dades assignades a cada \emph{cluster}.
	\end{itemize}
\end{itemize}

\subsection{Formalització}
\begin{itemize}
	\item Donat $\mathcal{P} = \{ \mu_1, ..., \mu_K \}$ recalcular els $\{r_{nk}\}$
	
	$$ r_{nk} =
	\begin{cases}
	1 & \text{si } k = argmin ||x_n - \mu_j||\ 1 \le j \le K \\ 0 & \text{en cas contrari}
	\end{cases}$$
	
	\item Recalcular $\mathcal{P}$ a partir dels $\{r_{nk}\}$
	
	\begin{itemize}
		\item $\frac{\partial J}{\partial \mu_k} = \sum_{n=1}^N r_{nk} \underbrace{\frac{\partial ||x_n - \mu_k||^2}{\partial \mu_k}}_{-2(x_n - \mu_k)} = 2\sum_{n=1}^N r_{nk} (\mu_k - x_n) = 0$
		
		$\sum_{n=1}^N r_{nk}\mu_k = \sum_{n=1}^N r_{nk}x_n \implies \mu_k\sum_{n=1}^N r_{nk} = \sum_{n=1}^N r_{nk}x_n \implies \boxed{\mu_k = \frac{\sum_{n=1}^N r_{nk}}{\sum_{n=1}^N r_{nk}}}$
		
		\item Tenint en compte que $\frac{\partial ||\vec{z}||^2}{\partial \vec{z}} = \begin{pmatrix}
		2z_1 \\ 2z_2 \\ \vdots \\ 2z_d
		\end{pmatrix} = 2\vec{z}$
	\end{itemize}
\end{itemize}

\subsection{Algorisme}

Inicialitzar $\mathcal{P}$

\textbf{Repetir}:
\begin{itemize}
	\item Re-calcular els $r_{nk}$ usant $\mathcal{P}$
	\item Re-calcular $\mathcal{P}$ usant els $r_{nk}$
\end{itemize}
\textbf{fins que} convergeixi

\textbf{retornar} $\mathcal{P},r_{nk}$


\subsection{Comentaris}

\begin{enumerate}
	\item $\Downarrow$ Cal una inicialització de $\mathcal{P}$, que és un problema en sí mateix
	\item $\Downarrow$ L'assignació de dades als \emph{clusters} $\{r_{nk}\}$ és \textbf{binària} per tant és difícil.
	\item $\Downarrow$ Cal un procediment extern de determinació de $K$.
\end{enumerate}

\section{Barreges de Gaussianes (Mixture of Gaussians)}

Determinació de \textbf{funcions de densitat de probabilitat (pdf)}. Tenim una mostra de dades  $\{x_1,...,x_n\}$ i.i.d (simple) generar una pdf $p(x)$. Això és impossible a no ser que reduïm les possibilitats.

El que farem serà agafar unes quantes gaussianes i comprovar quin percentatge de la solució final representen.

% Figura 5

En dues dimensions es tenen unes dades podem trobar uns conjunts de gaussianes, es solaparan perquè les gaussianes tenen amplada infinit. Un cop trobades les ponderem i busquem la densitat final.

% Figura 6

Una barreja de Gaussianes és una pdf de la forma:

$$ p(x) = \sum_{k=1}^K \underbrace{\pi_k}_{\text{Coeficients}}·\underbrace{N(x, \mu_k, \Sigma_k)}_{\text{Components de la barreja}}, 0 \le \pi_k \le 1, \sum_{k=1}^K \pi_k = 1$$

\begin{align*}
	\{\pi_k, \mu_k, \Sigma_k\} & \text{  desconegudes, paràmetres} \\
	K & \text{  desconegudea (hiper-paràmetre)}
\end{align*}

A vegades generalitzar el problema ajudar a trobar solucions que a simple vista no es poden veure per casos particulars. 

En principi la distribució $p$ només pertany de les variables $x$. Creem una nova variable aleatòria $z$ que ens ajuda a veure el problema de manera més general.

$$ p(\underbrace{x}_{x \in \mathbb{R}^d}) \rightarrow p(x,z) = p(x|z)·p(z)$$

$$ z = \begin{pmatrix}
z1 \\ \vdots \\ z_K
\end{pmatrix} \text{Només una component de z és 1} $$

S'interpreta com que ha estat la coponent $k$ la que ha generat X. Aquestes $z$ s'anomenen \textbf{variables aleatòries latents o ocultes}.

\begin{description}
	\item[Distribució marginal sobre z] $p(z_k = 1) = \pi_k$
	
	$p(z) = \prod_{k=1}^K {\pi_k}^{z_k}$
	
	\item[Distribució condicional] de $x$ respecte $z$
	
	$p(x|z_k = 1) = N(x, \mu_k, \Sigma_k)$
	
	\item[Distribució marginal sobre $x$] ara eliminem la $z$
	
	$p(x) = \sum_z p(x|z)p(z) = \sum_{k=1}^K \pi_k N(x, \mu_k, \Sigma_k)$
	
	\item[Distribució condicional] de $z$ respecte $x$
	
	$p(z|x) = \frac{p(x|z)p(z)}{p(x)}$
\end{description}
\end{document}