\documentclass[a4paper]{article}

\usepackage{amsmath}
\usepackage{amssymb,amsfonts}
\usepackage[catalan]{babel} % Language 
\usepackage{fontspec}
\usepackage[margin=2cm]{geometry}
\usepackage{graphicx}

\title{Tema 3: Mètodes de \emph{clustering}}

\begin{document}
	
\maketitle

\textbf{Objectiu}: El nostre objectiu és trobar agrupacions naturals de dades. Un grup o subgrup de dades és un \emph{cluster}

Les observacions que pertanyen al mateix \emph{cluster} s'assemblen entre elles

Les observacions que \textbf{no} pertanyen al mateix \emph{cluster} no s'assemblen tant

\textbf{\emph{clustering}}

\begin{itemize}
	\item el \textbf{procés} de trobar els \emph{clusters} presents en les dades
	\item el \textbf{resultat} del procés.
\end{itemize}

% Figures 1 i 2

\subsection{Mètodes de \emph{clustering}}

Un tipus de mètode són els mètodes jeràrquics:
\begin{itemize}
	\item \textbf{divisius} es dediquen a separar les dades en dos grups recursivament fins que només queda un punt.
	\item \textbf{aglomeratius} es dediquen a agrupar les dades des d'un punt fins a obtenir el \emph{cluster} final.
\end{itemize}

% Figura 3

També hi ha els mètodes combinatoris. Hi ha un algoritme voraç amb una funció de qualitat.

Hi ha els algoritmes probabilístics. De quantes maneres es poden agrupar $N$ dades en $K$ \emph{clusters}?

$$ S(N, K) = \frac{1}{K!} \sum_{k=1}^K (-1)^{K-k} \begin{pmatrix}
K \\ k
\end{pmatrix} \text{, nº d'Stirling del 2n tipus} $$

$$ S(19,4) \simeq = 10^10 $$

$$ \underbrace{B(N)}_{\text{nº de Bell}} \sum_{K=1}^{N} S(N,K) \implies \text{ ex.: } B(79) \simeq 3.89·10^85 $$

Així doncs hi ha diferents mètodes de \emph{clustering} per trobar els grups.

\subsection{Algorisme de $k$-means}

Tenim una mostra $\mathcal{D} = \{ x_1,..., x_N \} , x_i \in \mathbb{R}, 1 \le i \le N$. Fixa't $K$ externament, escollim un conjunt de $K$ \textbf{prototips}:

$$ \mathcal{P} = \{ \mu_1, ..., \mu_K \}, \mu \in \mathbb{R}^d $$

\end{document}